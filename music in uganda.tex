\documentclass{article}
\begin{document}

\title{  MUSIC IN UGANDA.}

\author{by NSUBUGA FRANCIS}
\date{February 2018}
\maketitle
 Uganda, is now ranked number three (3) in Africa as far as music and entertainment is concerned. Uganda is home to over 65 different ethnic groups and tribes, and they form the basis of all indigenous music. The Baganda, being the most prominent tribe in the country, have dominated the culture and music of Uganda over the last two centuries. The other tribes have their own musical styles passed down since the 18th century. 

The first form of popular music to arise out of traditional music was the Kadongo Kamu style of music, which arose out of traditional Ganda music. From the 80 till early 90z,Kadongo Kamu was dominatet by musicians such as Peterson Mutebi, Dan Mugula, Sebaduka Toffa, Fred Sonko, Livingstone Kasozi, Fred Masagazi, Baligidde, Abuman Mukungu, Gerald Mukasa, Sauda Nakakaawa, Matia Luyima, Herman Basudde, Livingstone Kasozi and Paulo Kaffero  music genres drew from Kadongo Kamu, making it the most influential style of music in Uganda. In the late 90z, a new music genre called Bax Ragga was formed by Abdu Mulaasi Currently, because of the effects of globalization, Uganda, like most African countries, has seen a growth in modern audio production. This has led to the adoption of western music styles like Dancehall and Hip Hop.

\textbf{Traditional music:}
Uganda's tribes are diverse and spread evenly throughout the country. The divide between the Nilotic peoples and the Bantu peoples is evident, with most Nilotic tribes like the Acholi and the Langi found in the northern part of the country while the Bantu tribes like the Baganda are found mostly in the south of the country.

Tribal music in Uganda, like in most African regions, is mainly functional. This means that most music and music activities usually have specific functions related to specific festivities like marriage, initiation, royal festivals, harvests and the like. The music is performed by skilled tribesmen who are good at various instruments and well versed with the stylistic elements of the music of their tribe. 

Most music is geared for dancing in the community, hence most tribes have specific dances associated with their music. Call and response style of singing is common with the Bantu  and is the many ways vital information is passed on to the listeners . Uganda's most populär DJ, Erycom was the first Ugandan to own a Youtube channel and he's amongst the first two Ugandans to make Ugandan music circulate online digitally.
Ugandan popular music is part of the larger African popular music

\textbf{Baganda:}
The Baganda are found in the central region of Uganda and are the largest ethnic group in the country. The kingdom is ruled by a king, known as a Kabaka. The kabaka has traditionally been the main patron of the music of Buganda. Musical instruments include various forms of drums, making percussion an integral part of the music. 

The massive and sacred royal drums are just one of the many drum types. The engalabi is another common drum. It is a long round shaped drum  with a high pitched sound used in synchronization of both instruments and dances.  The drums are used in unison with various other melodic musical instruments ranging from chordophones like the ennanga harp and the entongoli lyre, lamellophones, aerophones, and idiophones and the locally made fiddle called kadingidi. The locally made xylophone, called amadinda, is one of the largest in sub saharan Africa. 

The Baganda have a variety of vibrant dances that go along with the elaborate instrumentation. The bakisimba dance is the most common and most performed. There are others like nankasa and the amaggunju. The amaggunju is an exclusive dance developed in the palace for the Kabaka.

\textbf{Popular Music}
 Because of Uganda's turbulent political history, there was never enough time for there to be a thriving pop music industry until relative peace was restored in the late 1980s. By then, musicians like Philly Lutaaya, Afrigo Band, and Elly Wamala were the few Ugandan acts to have had mainstream music success. Jimmy Katumba and his music group the Ebonies were also popular at this time, especially towards the 1990s.
The 1990s saw Uganda's love affair with Jamaican music begin when artists like Shanks Vivi Dee, Ragga Dee, and others were influenced by Jamaican superstars like Shabba Ranks. They imported the Ragga music culture into Uganda and, although they faced stiff competition from other African music styles and musicians at the time, in particular Soukous from Congo and Kwaito from South Africa, they formed the foundation of the pop music industry. But it was not until the 21st century when musicians like Chameleone emerged that a pop music scene really began.

Ever since then Ugandan music has greatly improved and is considered among the best in Africa at this time. There are alot of great musicians in uganda rightnow. The likes of Bebe Cool, Bobi Wine, Goodlife and many others that are considere among the best in the continent. 
I would like to acknowlged wikipedia for the research well done 

\end{document}

